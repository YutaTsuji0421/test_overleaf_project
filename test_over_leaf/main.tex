\documentclass[a4paper,twocolumn, twoside]{bxjsarticle}  
\usepackage{zxjatype} %日本語環境にするためのもの
\usepackage[ipa]{zxjafont} %日本語環境にするためのもの
\usepackage[dvipdfmx]{graphicx} %図を入れるためのパッケージ
%\usepackage{here} %図を指定した場所に入れることのできるパッケージ
\usepackage{amsmath} %数式環境用パッケージ
\usepackage{bm}%ベクトルを太字として表示するためのパッケージ
\usepackage{comment}%まとめてコメントアウトするためのパッケージ
\usepackage{braket}%ブラケットを使うためのパッケージ

\makeatletter %以下はsectionごとに式番号を振るためのやつ
\@addtoreset{equation}{section}
\def\theequation{\thesection.\arabic{equation}}
\makeatother %ここまで

\newcommand{\Slash}[1]{{\ooalign{\hfil/\hfil\crcr$#1$}}} %Feynmanのスラッシュ用

\title{\vspace{-4cm}rheaでのKID測定(仮)}
\author{辻 悠汰}
\date{first draft:2020/04/24 \\
      this version:2020/04/24}

\begin{document}

\begin{abstract}
rheaを使ったKIDの測定を行ったので、そのレポートの概要がここに来る。
\end{abstract}

\maketitle


%\twocolumn[
%\maketitle
%\begin{abstract}
%ここが概要
%\end{abstract}
%]

%\tableofcontents
%\clearpage

\section*{1節}
ぞうの卵はおいしいぞう。ぞうの卵はおいしいぞう。ぞうの卵はおいしいぞう。ぞうの卵はおいしいぞう。ぞうの卵はおいしいぞう。ぞうの卵はおいしいぞう。ぞうの卵はおいしいぞう。ぞうの卵はおいしいぞう。ぞうの卵はおいしいぞう。ぞうの卵はおいしいぞう。ぞうの卵はおいしいぞう。ぞうの卵はおいしいぞう。ぞうの卵はおいしいぞう。ぞうの卵はおいしいぞう。ぞうの卵はおいしいぞう。ぞうの卵はおいしいぞう。ぞうの卵はおいしいぞう。ぞうの卵はおいしいぞう。ぞうの卵はおいしいぞう。ぞうの卵はおいしいぞう。ぞうの卵はおいしいぞう。ぞうの卵はおいしいぞう。ぞうの卵はおいしいぞう。ぞうの卵はおいしいぞう。ぞうの卵はおいしいぞう。ぞうの卵はおいしいぞう。ぞうの卵はおいしいぞう。ぞうの卵はおいしいぞう。ぞうの卵はおいしいぞう。ぞうの卵はおいしいぞう。ぞうの卵はおいしいぞう。ぞうの卵はおいしいぞう。ぞうの卵はおいしいぞう。ぞうの卵はおいしいぞう。ぞうの卵はおいしいぞう。ぞうの卵はおいしいぞう。ぞうの卵はおいしいぞう。ぞうの卵はおいしいぞう。ぞうの卵はおいしいぞう。ぞうの卵はおいしいぞう。ぞうの卵はおいしいぞう。ぞうの卵はおいしいぞう。ぞうの卵はおいしいぞう。ぞうの卵はおいしいぞう。ぞうの卵はおいしいぞう。ぞうの卵はおいしいぞう。ぞうの卵はおいしいぞう。ぞうの卵はおいしいぞう。ぞうの卵はおいしいぞう。ぞうの卵はおいしいぞう。ぞうの卵はおいしいぞう。ぞうの卵はおいしいぞう。ぞうの卵はおいしいぞう。ぞうの卵はおいしいぞう。ぞうの卵はおいしいぞう。ぞうの卵はおいしいぞう。ぞうの卵はおいしいぞう。ぞうの卵はおいしいぞう。ぞうの卵はおいしいぞう。ぞうの卵はおいしいぞう。ぞうの卵はおいしいぞう。ぞうの卵はおいしいぞう。ぞうの卵はおいしいぞう。ぞうの卵はおいしいぞう。ぞうの卵はおいしいぞう。ぞうの卵はおいしいぞう。ぞうの卵はおいしいぞう。ぞうの卵はおいしいぞう。ぞうの卵はおいしいぞう。ぞうの卵はおいしいぞう。ぞうの卵はおいしいぞう。ぞうの卵はおいしいぞう。ぞうの卵はおいしいぞう。ぞうの卵はおいしいぞう。ぞうの卵はおいしいぞう。ぞうの卵はおいしいぞう。ぞうの卵はおいしいぞう。ぞうの卵はおいしいぞう。ぞうの卵はおいしいぞう。ぞうの卵はおいしいぞう。ぞうの卵はおいしいぞう。ぞうの卵はおいしいぞう。ぞうの卵はおいしいぞう。ぞうの卵はおいしいぞう。ぞうの卵はおいしいぞう。ぞうの卵はおいしいぞう。ぞうの卵はおいしいぞう。ぞうの卵はおいしいぞう。ぞうの卵はおいしいぞう。ぞうの卵はおいしいぞう。ぞうの卵はおいしいぞう。ぞうの卵はおいしいぞう。ぞうの卵はおいしいぞう。ぞうの卵はおいしいぞう。ぞうの卵はおいしいぞう。ぞうの卵はおいしいぞう。ぞうの卵はおいしいぞう。ぞうの卵はおいしいぞう。ぞうの卵はおいしいぞう。ぞうの卵はおいしいぞう。ぞうの卵はおいしいぞう。ぞうの卵はおいしいぞう。ぞうの卵はおいしいぞう。ぞうの卵はおいしいぞう。ぞうの卵はおいしいぞう。ぞうの卵はおいしいぞう。ぞうの卵はおいしいぞう。ぞうの卵はおいしいぞう。ぞうの卵はおいしいぞう。ぞうの卵はおいしいぞう。ぞうの卵はおいしいぞう。ぞうの卵はおいしいぞう。ぞうの卵はおいしいぞう。ぞうの卵はおいしいぞう。ぞうの卵はおいしいぞう。ぞうの卵はおいしいぞう。ぞうの卵はおいしいぞう。
\section*{2節}
ぞうの卵はおいしいぞう。ぞうの卵はおいしいぞう。ぞうの卵はおいしいぞう。ぞうの卵はおいしいぞう。ぞうの卵はおいしいぞう。ぞうの卵はおいしいぞう。ぞうの卵はおいしいぞう。ぞうの卵はおいしいぞう。ぞうの卵はおいしいぞう。ぞうの卵はおいしいぞう。ぞうの卵はおいしいぞう。ぞうの卵はおいしいぞう。ぞうの卵はおいしいぞう。ぞうの卵はおいしいぞう。ぞうの卵はおいしいぞう。ぞうの卵はおいしいぞう。ぞうの卵はおいしいぞう。ぞうの卵はおいしいぞう。ぞうの卵はおいしいぞう。ぞうの卵はおいしいぞう。ぞうの卵はおいしいぞう。ぞうの卵はおいしいぞう。ぞうの卵はおいしいぞう。ぞうの卵はおいしいぞう。ぞうの卵はおいしいぞう。ぞうの卵はおいしいぞう。ぞうの卵はおいしいぞう。ぞうの卵はおいしいぞう。ぞうの卵はおいしいぞう。ぞうの卵はおいしいぞう。ぞうの卵はおいしいぞう。ぞうの卵はおいしいぞう。ぞうの卵はおいしいぞう。ぞうの卵はおいしいぞう。ぞうの卵はおいしいぞう。ぞうの卵はおいしいぞう。ぞうの卵はおいしいぞう。ぞうの卵はおいしいぞう。ぞうの卵はおいしいぞう。ぞうの卵はおいしいぞう。ぞうの卵はおいしいぞう。ぞうの卵はおいしいぞう。ぞうの卵はおいしいぞう。ぞうの卵はおいしいぞう。ぞうの卵はおいしいぞう。ぞうの卵はおいしいぞう。ぞうの卵はおいしいぞう。ぞうの卵はおいしいぞう。ぞうの卵はおいしいぞう。ぞうの卵はおいしいぞう。ぞうの卵はおいしいぞう。ぞうの卵はおいしいぞう。ぞうの卵はおいしいぞう。ぞうの卵はおいしいぞう。ぞうの卵はおいしいぞう。ぞうの卵はおいしいぞう。ぞうの卵はおいしいぞう。ぞうの卵はおいしいぞう。ぞうの卵はおいしいぞう。ぞうの卵はおいしいぞう。ぞうの卵はおいしいぞう。ぞうの卵はおいしいぞう。ぞうの卵はおいしいぞう。ぞうの卵はおいしいぞう。ぞうの卵はおいしいぞう。ぞうの卵はおいしいぞう。ぞうの卵はおいしいぞう。ぞうの卵はおいしいぞう。ぞうの卵はおいしいぞう。ぞうの卵はおいしいぞう。ぞうの卵はおいしいぞう。ぞうの卵はおいしいぞう。ぞうの卵はおいしいぞう。ぞうの卵はおいしいぞう。ぞうの卵はおいしいぞう。ぞうの卵はおいしいぞう。ぞうの卵はおいしいぞう。ぞうの卵はおいしいぞう。ぞうの卵はおいしいぞう。ぞうの卵はおいしいぞう。ぞうの卵はおいしいぞう。ぞうの卵はおいしいぞう。ぞうの卵はおいしいぞう。ぞうの卵はおいしいぞう。ぞうの卵はおいしいぞう。ぞうの卵はおいしいぞう。ぞうの卵はおいしいぞう。

\end{document}